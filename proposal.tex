\documentclass[]{article}
\usepackage{amsmath}
\usepackage{listings}
\usepackage{amssymb}
\usepackage{graphicx}
\usepackage{indentfirst}
%\usepackage[colorinlistoftodos]{todonotes}



%opening
\title{Text-to-Image-to-Text Translation using Cycle Consistent Adversarial Networks}
\author{Jeremy Ma, Satya Krishna Gorti}
\date{}

\begin{document}

\maketitle

\begin{abstract}

\end{abstract}

\section{Introduction}

Text-to-Image synthesis is a challenging problem that has a lot of room for improvement considering the current state-of-the-art results. Synthesized images from existing methods give a rough sketch of the described image but fail to capture the true essence of what the text describes. The recent success of Generative Adversarial Networks \cite{goodfellow2014generative} indicate that they are a good candidate for the choice of architecture to approach this problem.
\\

However the very nature of this problem is such that a piece of text can map to multiple valid images. The lack of such a direct one-to-one mapping means that traditional conditional GANs \cite{mirza2014conditional} cannot be used directly. We draw our inspiration from the recent works of image-to-image translation \cite{liu2017unsupervised}\cite{zhu2017unpaired} where a cycle consistent GANs have been trained and achieved very impressive results.
\\

We believe that using a cycle GAN for text-to-image-to-text translation will generate better results than existing approaches and give more photo-realistic images. The added advantage of framing the problem in a cycle consistent manner would also mean that the architecture can not only be a text-to-image synthesizing network but also an image captioning network.
\\

Therefore we have two generators $G$ and $F$. We train a a mapping $G: T_{emb} \mapsto Y$ and inverse mapping $F: Y \mapsto T_{emb}$ in a cycle consistent manner, where $T_{emb}$ is an embedding for the text that describes an image. The generators $G$ and $F$ have their corresponding discriminator $D_g$ and $D_f$.

\section{Related work}


\section{Project plan}

\section{Nice-to-halves}

\bibliographystyle{plain}
\bibliography{ref}

\end{document}
